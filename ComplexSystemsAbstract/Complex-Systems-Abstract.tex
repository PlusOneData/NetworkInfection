% Options for packages loaded elsewhere
\PassOptionsToPackage{unicode}{hyperref}
\PassOptionsToPackage{hyphens}{url}
%
\documentclass[
]{article}
\usepackage{lmodern}
\usepackage{amssymb,amsmath}
\usepackage{ifxetex,ifluatex}
\ifnum 0\ifxetex 1\fi\ifluatex 1\fi=0 % if pdftex
  \usepackage[T1]{fontenc}
  \usepackage[utf8]{inputenc}
  \usepackage{textcomp} % provide euro and other symbols
\else % if luatex or xetex
  \usepackage{unicode-math}
  \defaultfontfeatures{Scale=MatchLowercase}
  \defaultfontfeatures[\rmfamily]{Ligatures=TeX,Scale=1}
\fi
% Use upquote if available, for straight quotes in verbatim environments
\IfFileExists{upquote.sty}{\usepackage{upquote}}{}
\IfFileExists{microtype.sty}{% use microtype if available
  \usepackage[]{microtype}
  \UseMicrotypeSet[protrusion]{basicmath} % disable protrusion for tt fonts
}{}
\makeatletter
\@ifundefined{KOMAClassName}{% if non-KOMA class
  \IfFileExists{parskip.sty}{%
    \usepackage{parskip}
  }{% else
    \setlength{\parindent}{0pt}
    \setlength{\parskip}{6pt plus 2pt minus 1pt}}
}{% if KOMA class
  \KOMAoptions{parskip=half}}
\makeatother
\usepackage{xcolor}
\IfFileExists{xurl.sty}{\usepackage{xurl}}{} % add URL line breaks if available
\IfFileExists{bookmark.sty}{\usepackage{bookmark}}{\usepackage{hyperref}}
\hypersetup{
  pdftitle={Integrating COVID Models at Different Scales for Infection Risk Estimation},
  hidelinks,
  pdfcreator={LaTeX via pandoc}}
\urlstyle{same} % disable monospaced font for URLs
\usepackage[margin=1in]{geometry}
\usepackage{graphicx,grffile}
\makeatletter
\def\maxwidth{\ifdim\Gin@nat@width>\linewidth\linewidth\else\Gin@nat@width\fi}
\def\maxheight{\ifdim\Gin@nat@height>\textheight\textheight\else\Gin@nat@height\fi}
\makeatother
% Scale images if necessary, so that they will not overflow the page
% margins by default, and it is still possible to overwrite the defaults
% using explicit options in \includegraphics[width, height, ...]{}
\setkeys{Gin}{width=\maxwidth,height=\maxheight,keepaspectratio}
% Set default figure placement to htbp
\makeatletter
\def\fps@figure{htbp}
\makeatother
\setlength{\emergencystretch}{3em} % prevent overfull lines
\providecommand{\tightlist}{%
  \setlength{\itemsep}{0pt}\setlength{\parskip}{0pt}}
\setcounter{secnumdepth}{-\maxdimen} % remove section numbering

\title{Integrating COVID Models at Different Scales for Infection Risk
Estimation}
\author{}
\date{\vspace{-2.5em}}

\begin{document}
\maketitle

COVID-19 infections result from interactions that happen at multiple
spatial and temporal scales. When working at population level spatial
and temporal scales, it is feasible to model disease transmission
systems using systems of ordinary differential equations. As spatial and
temporal scales become smaller, the structure of social and physical
interactions become more influential and stochastic events become more
important. Appropriately integrating the interactions between processes
that occur across spatial and temporal scales is essential for
simulating systems of disease transmission and understanding infection
risk.

We have developed a simulation engine that is capable of integrating a
spatially explicit COVID-19 case estimation technique at the county
scale with institution level disease transmission at a ``building''
scale. The case estimation technique takes into account location
specific factors around infection control and population level movement
to estimate disease burden in a given location. The institution level
model uses a multigraph to integrate social and spatial contact networks
under various hazard reduction strategies at two different time scales.
This allows us to model individual level interactions in the local
context of the COVID-19 pandemic.

Using our model, we were able to recover outbreak behavior in multiple
systems\textbf{Validation} Need examples of known infection networks -
summer camps, cruise ship, uss theo rosevelt nba, nfl?, dinner parties?,
schools?, hospitals? \textbf{Validation} . - need to be able to
implement interventions at a specific time point

We have demonstrated that our model provides realistic estimates of the
COVID-19 outbreak size and spread. By integrating models at multiple
scales, our simulation engine empowers decision makers to develop
location specific preparedness policies based on realistic estimates of
how COVID will spread through their institutions. Because our framework
is highly extensible, we plan to add vaccination modules as that
information becomes available.

\begin{verbatim}
## [1] "infected"    "recovered"   "susceptible" "leave"
\end{verbatim}

\includegraphics{Complex-Systems-Abstract_files/figure-latex/unnamed-chunk-1-1.pdf}

\end{document}
